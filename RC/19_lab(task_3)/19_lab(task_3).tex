\documentclass[a4paper, 12pt]{article}%тип документа

%отступы
\usepackage[left=2cm,right=2cm,top=2cm,bottom=3cm,bindingoffset=0cm]{geometry}

%Русский язык
\usepackage[T2A]{fontenc} %кодировка
\usepackage[utf8]{inputenc} %кодировка исходного кода
\usepackage[english,russian]{babel} %локализация и переносы

%Вставка картинок
\usepackage{wrapfig}
\usepackage{graphicx}
\graphicspath{{pictures/}}
\DeclareGraphicsExtensions{.pdf,.png,.jpg}

%оглавление
\usepackage{titlesec}
\titlespacing{\chapter}{0pt}{-30pt}{12pt}
\titlespacing{\section}{\parindent}{5mm}{5mm}
\titlespacing{\subsection}{\parindent}{5mm}{5mm}
\usepackage{setspace}

%Графики
\usepackage{multirow}
\usepackage{pgfplots}
\pgfplotsset{compat=1.9}

%Математика
\usepackage{amsmath, amsfonts, amssymb, amsthm, mathtools}

%Стиль страницы
\usepackage{fancyhdr}
\pagestyle{fancy}

\begin{document}

\begin{titlepage}

\begin{center}
%\vspace*{1cm}
\large\textbf{Московский Физико-Технический Институт}\\
\large\textbf{(государственный университет)}
\vfill
\line(1,0){430}\\[1mm]
\huge\textbf{Работа 19}\\
\huge\textbf{Задание 3}\\
\line(1,0){430}\\[1mm]
\vfill
\large Сибгатуллин Булат, ФРКТ\\
\end{center}

\end{titlepage}
\fancyhead[L] {Работа 19 (задание 3)}
\noindent \textbf{Цель работы: познакомиться с поведением фильтров Чебышева и Баттерворта, эллпитическим фильтром и посмотреть, как они ведут себя при преобразовании в полосовые фильтры.} \\
\indent text\\
\noindent \textbf{В работе используются: matlab v.8 для анализа поведения фильтров.} \\
\indent text

\section*{Теория}

Фильтр Баттерворта порядка n получается при выборе:

%Дописать теорию

\section*{Выполнение работы}

\begin{enumerate}

\item Посмотрим на поведение фильтра Баттервота при увеличении количества полюсов (n). На АЧХ можем видеть, что скорость затухания на декаду увеличивается пропорционально $20\cdot n$, что совпадает с теорией. Также можем заметить, что когда количество полюсов нечетное - один из них лежит на вещественной оси, а когда четное - нет.

Перейдем к фильтру Чебышева. Заметим, что при увеличении количества полюсов они начинают сдвигаться в сторону мнимой оси на комплексной плоскости. На АЧХ с увеличением полюсов увеличивается количество колебаний в полосе пропускания. При уменьшении параметра неравномерности ($\varepsilon$), ширина полосы неравномерности (на АЧХ) уменьшается, если для $\varepsilon = 1$ полоса имела ширину в 3dB, то при $\varepsilon = 0.1$ полоса имеет ширину $\ll 1$ dB. Также при уменьшении $\varepsilon$ расстояние между полюсами и мнимой осью в комплескной плоскости увеличивается.

Наконец рассмотрим эллиптический фильтр. При увеличении количества полюсов они смещаются в сторону мнимой оси в комплексной плоскости, а на самой мнимой оси появляются нули. На АЧХ увеличивается величина затухания на декаду. При уменьшении неравномерности уменьшается скорость затухания на декаду (график на АЧХ сужается), а на комплексной плоскости расстояние между полюсами и мнимой осью увеличивается. При уменьшении селективности ($\eta$) график сужается к оси y, а на комплексной плоскости нули сдвигаются в сторону вещественной оси. Также можем подтвердить, что при нечетном количестве полюсов эллиптический фильтр имеет нули в бесконечности, а при четном, нет.

\item Найдем уровень затухания фильтра Чебышева при параметрах $n = 7, \varepsilon = 1, \eta = 2$:

\[20 \log_10 \mid H(s)\mid = -74 dB\]

Тот же уровень затухания достигается у фильтра Баттерворта с параметрами $n = 7, \eta = 3.4$.

\item Уровень затухания фильтра Чебышева с параметрами $n = 7, \varepsilon = 1, \eta = 1.5$ будет равень $stoplevel \simeq 421.5 (52.5 dB)$. Такое же затухание будет у фильтра Баттерворта при $n = 15, \eta = 1.5$.

\item Уровень затухания эллиптического фильтра с параметрами $n = 7, \varepsilon = 1, \eta = 1.1$ будет равен $stoplevel \simeq 608.5 (55.7 dB)$. Такое же значение затухания достигается фильтром Чебышева с $n = 7, \varepsilon = 1$ и селективностью $\eta = 1.1$.

\item Определим максимальные добротности полюса полосовых фильтров Баттерворта и Чебышева с $Q = 10, n = 9, \varepsilon = 1$. Для преобразованного фильтра Чебышева $Q_{max} = 357.9$, а для преобразованного фильтра Баттерворта $Q_{max} = 57.7$.

\item Вычислим созвездия и характеристики эллиптического фильтра $ellp(7, 1, 1.5)$. Преборазуем его в филтр верхних частот с параметрами $n = 7, \varepsilon = 1, \eta = 1.5$ и полосовой фильтр с параметрами $n = 7, \varepsilon = 1, \eta = 1.5$ и меняющейся добротностью $Q = 2;5;10$. Для всех Q измерим максимальные добротности полосового фильтра:

\begin{center}
\begin{tabular}{|c|c|c|c|c|}
\hline 
 & Фильтр верхних частот & \multicolumn{3}{c|}{Полосовой фильтр} \\ 
\hline 
Q &  & 2 & 5 & 10 \\ 
\hline 
$Q_{max}$ & 23.48 & 97.97 & 238.94 & 476.14 \\ 
\hline 
\end{tabular} 
\end{center}

\item Возьмем полосовой фильтр с параметрами $Q = 20, \varepsilon = 1, \eta_1 = 10^4 \: (80dB)$. Оценим селективность $\eta$, которую обеспечивает эллиптический фильтр порядка $n = 7$ c таким затуханием: $\eta = 1.36$. Подберем порядок фильтра Чебышева, который обеспечит сопоставимое с ним значение селективности: $n = 12$.

Преобразуем эти фильтры в полосовые с $Q = 20$, сравним максимальные добротности полюсов:

\begin{center}
\begin{tabular}{|c|c|c|}
\hline 
 & Фильтр Чебышева & Фильтр Баттерворта \\ 
\hline 
$Q_{max}$ & 2084.96 & 1049.39 \\ 
\hline 
\end{tabular} 
\end{center}

\end{enumerate}

\end{document}